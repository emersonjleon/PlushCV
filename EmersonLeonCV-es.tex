%%%%%%%%%%%%%%%%%%%%%%%%%%%%%%%%%%%%%%%
% PlushCV - One Page Two Column Resume
% xelatex EmersonLeonCV-es.tex&
%
% LaTeX Template
% Version 1.0 (11/28/2021)
%
% Author:
% Shubham Mazumder (http://mazumder.me)
%
% Hacked together from:
% https://github.com/deedydas/Deedy-Resume
%
% IMPORTANT: THIS TEMPLATE NEEDS TO BE COMPILED WITH XeLaTeX
%
% 
%%%%%%%%%%%%%%%%%%%%%%%%%%%%%%%%%%%%%%
% 
% TODO:
% 1. Figure out a smoother way for the document to flow onto the next page.
% 3. Add more icon options 
% 4. Fix hacky left alignment on contact line
% 5. Remove Hacky fix for awkward extra vertical space
% 
%%%%%%%%%%%%%%%%%%%%%%%%%%%%%%%%%%%%%%
%
% CHANGELOG:
%
%%%%%%%%%%%%%%%%%%%%%%%%%%%%%%%%%%%%%%%
%
% Known Issues:
% 1. Overflows onto second page if any column's contents are more than the vertical limit.
%%%%%%%%%%%%%%%%%%%%%%%%%%%%%%%%%%%%%%
%%Icons:
%%Main: https://icons8.com/icons/carbon-copy
%%%%%%%%%%%%%%%%%%%%%%%%%%%%%%%%%%

\documentclass[]{plushcv}
\usepackage{fancyhdr}
\pagestyle{fancy}
\fancyhf{}
\begin{document}

%%%%%%%%%%%%%%%%%%%%%%%%%%%%%%%%%%%%%%
%
%     TITLE NAME
%
%%%%%%%%%%%%%%%%%%%%%%%%%%%%%%%%%%%%%%

\namesection{Emerson Le\'on}{}{Matemático/ Desarrollador de Software }{\mycontactline{\href{https://emersonjleon.pythonanywhere.com/}{emersonjleon.pythonanywhere.com}}{\href{https://www.github.com/emersonjleon}{emersonjleon}}{\href{https://www.linkedin.com/in/emersonjleon}{emersonjleon}}{\href{mailto:emersonleon@gmail.com}{emersonleon@gmail.com}}{\href{tel:3057236943}{+573057236943}}} %4915754767896

%\namesection{}{Emerson Le\'on}{Mathematician doing Programming, Visualization and Data Science}{\mycontactline{\href{https://emersonjleon.pythonanywhere.com/}{emersonjleon.pythonanywhere.com}}{\href{https://www.linkedin.com/in/emersonjleon}{emersonjleon}}{\href{mailto:emersonleon@gmail.com}{emersonleon@gmail.com}}{\href{tel:+4915754767896}{+49 157 54767896}}}


% \namesection{Firstname}{Lastname}{Full Stack Software Engineer}{\contactline{\href{https://www.mazumder.me}{mazumder.me}}{\href{https://www.github.com/sansquoi}{sansquoi}}{\href{https://www.linkedin.com/mazumders}{mazumders}}{\href{mailto:shubham.mazumder@gmail.com}{first.last@email.com}}{\href{tel:+1999999999}{9999999999}}}

%%%%%%%%%%%%%%%%%%%%%%%%%%%%%%%%%%%%%%
%
%     COLUMN ONE
%
%%%%%%%%%%%%%%%%%%%%%%%%%%%%%%%%%%%%%%

\begin{minipage}[t]{0.70\textwidth} 


%%%%%%%%%%%%%%%%%%%%%%%%%%%%%%%%%%%%%%
%     EXPERIENCE
%%%%%%%%%%%%%%%%%%%%%%%%%%%%%%%%%%%%%%

\section{Experiencia}
\runsubsection{Universidad Antonio Nariño}
\descript{| Coordinador Olimpiada de Inteligencia Artificial}
\location{Junio 2022 – Julio 2023| Bogotá , Colombia}
\vspace{\topsep} % Hacky fix for awkward extra vertical space
\begin{tightemize}
\sectionsep
\item Proyecto con Tercer Piso Danza, convocatoria de \emph{Idartes} para arte, ciencia y tecnología.
\item Creé una página para generar historias con inteligencia artificial, usando GPT-3 y Stable Diffusion desde OpenAI \location{\href{https://elcyborgchaman.pythonanywhere.com}{https://elcyborgchaman.pythonanywhere.com}}
\item Habilidades: desarrollo web · arte and ciencia · creatividad · python · openAI · flask · django · javaScript  · inteligencia artificial  · three.js  ·  sqlAlchemy · html · css% · relational databases 
\end{tightemize}
\sectionsep



\runsubsection{El Cyborg Cham\'an}
\descript{| Desarrollo de Software}
\location{Junio 2022 – Julio 2023| Bogotá , Colombia}
\vspace{\topsep} % Hacky fix for awkward extra vertical space
\begin{tightemize}
\sectionsep
\item Proyecto con Tercer Piso Danza, convocatoria de \emph{Idartes} para arte, ciencia y tecnología.
\item Creé una página para generar historias con inteligencia artificial, usando GPT-3 y Stable Diffusion desde OpenAI \location{\href{https://elcyborgchaman.pythonanywhere.com}{https://elcyborgchaman.pythonanywhere.com}}
\item Habilidades: desarrollo web · arte and ciencia · creatividad · python · openAI · flask · django · javaScript  · inteligencia artificial  · three.js  ·  sqlAlchemy · html · css% · relational databases 
\end{tightemize}
\sectionsep

%% \descript{ Countryside Experience}
%% \location{July  2017 - Apr 2022| Tena, Cundinamarca, Colombia}

\sectionsep
\runsubsection{Universidad de los Andes}
\descript{| Investigador posdoctoral | Profesor}
\location{Julio 2015 – Junio 2017 | Bogotá, Colombia}
\begin{tightemize}
  \sectionsep
\item Investigación en geometría discreta, politopos, combinatoria.
\item Clases: Álgebra lineal, cálculo diferencial, integral and multivariado, geometría discreta
%% \item Organizing: Ehrhart theory seminar
\item Habilidades:  geometría discreta · geometría computacional  · enseñanza · educación  · linux · desarrollo web · html · modelos matemáticos · sagemath% · julia 
\end{tightemize}
\sectionsep

\runsubsection{Olimpiada Colombiana de Matemáticas}
\descript{| Comité Organizador | Profesor }
\location{Junio 2002 – Diciembre 2006 | Bogotá, Colombia}
\begin{tightemize}
  \sectionsep
\item Coorganizador de competencias matemáticas anuales a nivel nacional
\item Ayudé a preparar jovenes colombianos con talento en la resolución de problemas matemáticos para participar en competencias internacionales.
  
\item Habilidades: enseñanza matemática  · organización de eventos  · resolución de problemas  · diseño de nuevos problemas · álgebra  · teoría de números · geometría  · combinatoria
\end{tightemize}
%\sectionsep


%% \runsubsection{Chess Engine}
%% \descript{| C++}
%% \location{2018}
%% \begin{tightemize}
%% \item \emph{Combinatorial Relationship Between Finite Fields and Fixed Points of Functions Going Up and Down}, Emerson Le\'on, Juli\'an Pulido,  \url{https://arxiv.org/abs/2111.13745}), preprint, 2021
%% \end{tightemize}
%% \sectionsep

%% \runsubsection{Speech-enabled Chatbot}
%% \descript{| C\#, Microsoft Bot Framework}
%% \location{2018}
%% \begin{tightemize}
%% \item \emph{Binomial Inequalities of Chromatic, Flow, and Ehrhart Polynomials},  Matthias Beck, Emerson Le\'on, {Discrete \& Computational Geometry}, 2021, 
%% {\bf 66}, no. 2, 464–474, 2021. \href{https://arxiv.org/abs/1804.00208}{\texttt{https://arxiv.org/abs/1804.00208}}.



%%%%%%%%%%%%%%%%%%%%%%%%%%%%%%%%%%%%%%
%     Education
%%%%%%%%%%%%%%%%%%%%%%%%%%%%%%%%%%%%%%


\section{Education} 
\runsubsection{Freie Universität, Berlin}
\descript{| Doctor rer. nat. Mathematics}
\location{Jan 2007 - Feb 2015 |
 advisor: Günter M. Ziegler }
\begin{tightemize}
\item Research in discrete geometry, mathematical modeling, topological spaces, moduli spaces, multidimensional geometry.
\item In the thesis we defined and analyzed a mathematical object that parameterized all possible ways to divide the d-dimensional space into n convex regions.
  \item Scholarship Phase I and Phase II by Berlin Mathematical School.\end{tightemize}
\sectionsep

\runsubsection{Universidad Nacional de Colombia, Bogotá}
\descript{| Bachelor's in Mathematics}
\location{Jul 2001 - Sep 2006}
\begin{tightemize}
  %\item T
\item Tuition reimbursement due to grades on the top 2 each semester.\\
\end{tightemize}

%\location{ Cum. GPA: 3.7 / 4.0 }
%% \sectionsep

\runsubsection{Other courses}
%% \descript{| Bachelor's in Mathematics}
\location{}
\begin{tightemize}
  %\item T
\item Kaggle: pandas, machine learning (basic, intermediate, explainability)
\end{tightemize}

%\location{ Cum. GPA: 3.7 / 4.0 }

\sectionsep
\runsection{Visualization projects}
%% \runsubsection{3D visualization}
\descript{| python | javaScript | three.js}
%\location{}
%% \vspace{\topsep} % Hacky fix for awkward extra vertical space
\begin{tightemize}
\sectionsep
\item  \href{https://emersonjleon.pythonanywhere.com/visualization}{Spline surfaces}: Some surfaces generated by piecewise polynomial equations for its coordinates. Software experiments for research in spline theory \location{\href{https://emersonjleon.pythonanywhere.com/visual}{link}} 
%% \end{tightemize}


%% \location{3D chip firing}
%% \begin{tightemize}
\item 3D chip firing: Code that generates three-dimensional chip firing configurations and help to visualize them online, with various customizable parameters \location{\href{https://emersonjleon.pythonanywhere.com/chipfiring}{link}} 
\end{tightemize}
%% \sectionsep


%%%%%%%%%%%%%%%%%%%%%%%%%%%%%%%%%%%%%%
%     AWARDS
%%%%%%%%%%%%%%%%%%%%%%%%%%%%%%%%%%%%%%

% \section{Awards} 
% \begin{tabular}{rll}
% 2020	     & Finalist & Lorem Ipsum\\
% 2018	     & $2^{nd}$ & Dolor Sit Amet\\
% 2015	     & Finalist  & Cras posuere\\
% \\
% \end{tabular}
% \sectionsep
%%%%%%%%%%%%%%%%%%%%%%%%%%%%%%%%%%%%%%
%
%     COLUMN TWO
%
%%%%%%%%%%%%%%%%%%%%%%%%%%%%%%%%%%%%%%

\end{minipage} 
\hfill
\begin{minipage}[t]{0.25\textwidth} 

%%%%%%%%%%%%%%%%%%%%%%%%%%%%%%%%%%%%%%
%     SKILLS
%%%%%%%%%%%%%%%%%%%%%%%%%%%%%%%%%%%%%%

\section{Skills}
\subsection{Programming}
\location{Proficient:}
python \textbullet{} javaScript  \textbullet{} html \textbullet{} css \textbullet{} sagemath \textbullet{} \LaTeX\ \\ 
%\sectionsep
%% \location{Experienced:}
%%  \textbullet{} CSS \textbullet{}  C++  \\
%% \sectionsep
\location{Familiar:}
 C++ \textbullet{}  matlab \textbullet{} Julia %\textbullet{} polymake \\
% R
 \sectionsep

 \subsection{Libraries/Frameworks}
%% \sectionsep
flask \textbullet{} django \textbullet{} three.js \textbullet{} openAI \textbullet{} pandas \textbullet{} sklearn  \textbullet{} matplotlib \textbullet{} pygame \textbullet{}   sqlalchemy \textbullet{}  node.js  \textbullet{} tensorflow \textbullet{} sagemath\\
\sectionsep
%% \sectionsep
%% \subsection{Tools/Platforms}
%% \sectionsep
%% Git \textbullet{} Gulp \textbullet{} Webpack \textbullet{} Heroku    \\ Wordpress \textbullet{} Docker \\
\subsection{Languages}
%\sectionsep
Spanish \textbullet{} English \textbullet{} German 

%% \sectionsep

%%%%%%%%%%%%%%%%%%%%%%%%%%%%%%%%%%%%%%
%     Publications
%%%%%%%%%%%%%%%%%%%%%%%%%%%%%%%%%%%%%%
\section{Publications}


\textbullet{} \href{https://arxiv.org/abs/1804.00208}{Binomial Inequalities of Chromatic, Flow, and Ehrhart Polynomials}, with  Matthias Beck \location{Disc. Comp. Geometry, 2021}

\textbullet{} \href{https://arxiv.org/abs/1511.02904}{Spaces of Convex n-Partitions}
\location{New trends in intuitive Geometry, 2017}

\textbullet{} \href{https://arxiv.org/abs/2111.13745}{Combinatorial Relationship Between Finite Fields and Fixed Points of Functions Going Up and Down} \location{preprint, 2023}


\sectionsep

%%%%%%%%%%%%%%%%%%%%%%%%%%%%%%%%%%%%%%
%     AWARDS
%%%%%%%%%%%%%%%%%%%%%%%%%%%%%%%%%%%%%%

\section{Awards} 
\subsection{Silver+2 Bronce Medals}
\descript{International Math Olympiad IMO}
\location{Rumania 1999 |  Korea 2000 |  USA 2001 }


\subsection{Gold Medal}
\descript{Iberoamerican Math Olympiad} First place
\location{Cuba  1999 }

%% \location{ advisor: Günter M. Ziegler }

%% \sectionsep
%% \subsection{Universidad Nacional de Colombia, Bogotá}
%% \descript{Bachelor's in Mathematics}
%% \location{Jul 2001 | Sep 2006}
%% School of Computing \\
%% %\location{ Cum. GPA: 3.7 / 4.0 }
%% \sectionsep

% %%%%%%%%%%%%%%%%%%%%%%%%%%%%%%%%%%%%%%
% %     REFERENCES
% %%%%%%%%%%%%%%%%%%%%%%%%%%%%%%%%%%%%%%


%% luisgi
\section{References} 
\href{https://serrano.academy/}{\textbf{Luis Serrano}, AI scientist, Serrano Academy}
\begingroup
\setbox0=\hbox{
\includegraphics[scale=0.1,trim={0 1cm 0cm 0cm}]{icons/main/mail.png}\hspace{0.3cm} luisgui.serrano@gmail.com
}
\parbox{\wd0}{\box0}
\endgroup

%%%%%not before
%% \begingroup
%% \setbox0=\hbox{
%% \includegraphics[scale=0.1,trim={0 1.25cm -0.4cm 0cm}]{icons/main/phone.png}\hspace{0.3cm}+19999999999
%% }
%% \parbox{\wd0}{\box0}\endgroup


%\\
%%%%%%%%%%%%%%%fede
\sectionsep
\href{https://fardila.com/}{\textbf{Federico Ardila}}, Math Professor, San Francisco State University 
\\
\begingroup
\setbox0=\hbox{
\includegraphics[scale=0.1,trim={0 1cm 0cm 0cm}]{icons/main/mail.png}\hspace{0.3cm} federico@sfsu.edu 
}
\parbox{\wd0}{\box0}
\endgroup



%% \begingroup
%% \setbox0=\hbox{
%% \includegraphics[scale=0.1,trim={0 1.25cm -0.4cm 0cm}]{icons/main/phone.png}\hspace{0.3cm}+19999999999
%% }
%% \parbox{\wd0}{\box0}\endgroup
%\\

%%%%%%%%%%%%%%%%%%%%%%%%%%%%%%%%%%%%%%
%     COURSEWORK
%%%%%%%%%%%%%%%%%%%%%%%%%%%%%%%%%%%%%%

%% \section{Teaching}

%% \subsection{Graduate}
%% Discrete geometry \\%\textbullet{}\\
%% Ehrhart theory seminar

%% \subsection{Undergraduate}

%% Linear algebra \\%\textbullet{}\\
%% Differential calculus \\%\textbullet{}\\
%% Integral calculus \\%\textbullet{}\\
%% Multivariate calculus \\%\textbullet{}\\

%% \textbullet{}\\ 
 %% \\

\end{minipage} 
\end{document}  \documentclass[]{article}

